%% MARK: Class Options
% Flags for packages
\bool_new:N	\g_coursework_loadcode_bool	% Algorithm/Source code formatting
\bool_new:N	\g_coursework_loaddiag_bool	% TikZ, pgfplots, etc

% Set flags to FALSE by default
\bool_gset_false:N	\g_coursework_loadcode_bool
\bool_gset_false:N	\g_coursework_loaddiag_bool

% Provide class options to load packages
\DeclareOption{code}{\bool_gset_true:N \g_coursework_loadcode_bool}
\DeclareOption{diagram}{\bool_gset_true:N \g_coursework_loaddiag_bool}

% Font
\bool_new:N \g_coursework_cmufont_bool
\bool_gset_false:N \g_coursework_cmufont_bool
\DeclareOption{cmu}{
	\bool_gset_true:N \g_coursework_cmufont_bool
}

% Pass all remaining options to parent class
\DeclareOption*{
	\PassOptionsToClass{\CurrentOption}{\c__coursework_class_parent_class_tl}
}
\ProcessOptions\relax

%% MARK: Load Parent Class
\LoadClass[letterpaper, 10pt]{\c__coursework_class_parent_class_tl}

%% Load Common Packages
\RequirePackage[
	top = 1.45in,
	bottom = 1.35in,
	inner = 1.5in,
	outer = 2in,
	marginparwidth=1.4in,
	marginparsep=0.25in
]{geometry}
\RequirePackage{framed}
\RequirePackage{fancyhdr}
\RequirePackage{titlesec}
\RequirePackage[overload]{textcase}
\RequirePackage[titles]{tocloft}
\RequirePackage{fancyhdr}
\RequirePackage[side,flushmargin]{footmisc}
\RequirePackage{enumitem}
\RequirePackage{hyperref}
\RequirePackage{xcolor}
\RequirePackage{xpatch}
\RequirePackage{graphicx}
\RequirePackage{changepage}

% Mathematics
\RequirePackage{amsmath}
\RequirePackage{amssymb}
\RequirePackage{amsthm}
\RequirePackage{thmtools}
\RequirePackage{mathtools}
\RequirePackage{bm}
\RequirePackage{mdframed}

% Physics
\RequirePackage{physics}
\RequirePackage{siunitx}
\RequirePackage{tensor}

% Chemistry
\RequirePackage[version=4]{mhchem}

% Computer Science
\bool_if:NT	\g_coursework_loadcode_bool {
	\RequirePackage{minted} \usemintedstyle{tango}
	\RequirePackage{algorithm}
	\RequirePackage{algpseudocode}
}

% Diagrams
\bool_if:NT	\g_coursework_loaddiag_bool {
	\RequirePackage{tikz}
	\RequirePackage{pgfplots}
	\RequirePackage[american, siunitx]{circuitikz}
	\RequirePackage{tikz-feynman}
	\RequirePackage{chemfig}

	\ctikzset{bipoles/thickness=1}
	\pgfplotsset{compat=1.15}
}

%% MARK: Fonts
\ifPDFTeX
	% Allow UTF-8 characters
	\RequirePackage[utf8]{inputenc}
	\RequirePackage[T1]{fontenc}
	
	% Adjust kerning
	\RequirePackage[letterspace=100]{microtype}

	% Create empty macro for kerning
	\NewDocumentCommand\widekern{ } {\lsstyle}

	% Load pdflatex Palatino
	\bool_if:NTF \g_coursework_cmufont_bool {\RequirePackage{lmodern}} {\RequirePackage{libertineRoman}}

	\def\ipafont\relatex
	\NewDocumentCommand\ipafont{ }{}
	\NewDocumentCommand\setipafont{ m }{}
	\NewDocumentCommand\ipa{ m }{}
\else
	% Load fontspec
	\RequirePackage[no-math]{fontspec}

	% Wide kerning for title page / sections
	\bool_if:NTF \g_coursework_cmufont_bool {
		\NewDocumentCommand\widekern{ } {\addfontfeatures{LetterSpace=8}}
		\setmainfont[
			ItalicFont={CMUSerif-Italic},
			BoldFont={CMUSerif-Bold},
			BoldItalicFont={CMUSerif-BoldItalic},
			SlantedFont={CMUSerif-RomanSlanted},
			BoldSlantedFont={CMUSerif-BoldSlanted},
			Ligatures={Common, Discretionary, Contextual},
		]{CMUSerif-Roman}
	} {
		\NewDocumentCommand\widekern{ } {\addfontfeatures{LetterSpace=10}}
		\setmainfont[
			ItalicFont={LinLibertineOI},
			BoldFont={LinLibertineOB},
			BoldItalicFont={LinLibertineOBI},
			Ligatures={Common, Discretionary, Contextual},
		]{LinLibertineO}
	}

	\setmonofont[
		ItalicFont={CMUTypewriter-Italic},
		BoldFont={CMUTypewriter-Bold},
		BoldItalicFont={CMUTypewriter-BoldItalic},
		SlantedFont={CMUTypewriter-Oblique}
	]{CMUTypewriter-Regular}

	% Define ipafont
	\newfontfamily\ipafont{LinLibertineO}[
			BoldFont={LinLibertineOB},
			BoldItalicFont={LinLibertineOBI},
			Ligatures={Common, Discretionary, Contextual},
	]
	\NewDocumentCommand\setipafont{ m }{
		\newfontfamily\ipafont{#1}
	}
	\NewDocumentCommand\ipa{ m }{{\ipafont #1}}
\fi

%% Define Linguistics commands
\NewDocumentCommand\phon{ m }{$/${\ipafont #1}$/$}
\NewDocumentCommand\allo{ m }{$[${\ipafont #1}$]$}
\NewDocumentCommand\orth{ m }{$\langle${\ipafont #1}$\rangle$}
\NewDocumentCommand\where{ }{$\big/$}

% Define line spacing
\NewDocumentCommand\normalspacing{ } {\renewcommand{\baselinestretch}{1.2}}
\NewDocumentCommand\tightspacing{ }  {\renewcommand{\baselinestretch}{0.8}}
\normalspacing

% Define Colors
% Link colors
\definecolor{linkblue}{HTML}{2364AA}
\definecolor{linkred}{HTML}{B52639}
\definecolor{linkgold}{HTML}{C1900B}
\definecolor{yaleblue}{HTML}{00356B}
\definecolor{backgroundgold}{rgb}{1,0.93,0.8}
\definecolor{backgroundgrey}{rgb}{0.95,0.95,0.95}

\hypersetup{
  linkcolor  = linkgold,
  citecolor  = linkgold,
  urlcolor   = linkgold,
  colorlinks = true,
}

%% Define custom commands
% Footnotes
\let\oldfootnote\footnote
\RenewDocumentCommand\footnote{ m } {\oldfootnote{\textit{#1}}}

% Margin Notes
% This is used instead of marginpar
% to correct the placement at the
% beginning of environments.
\NewDocumentCommand\note{ m }{
	\begingroup
	\renewcommand\thefootnote{}\footnote{#1}%
  	\addtocounter{footnote}{-1}
  	\endgroup
}

% Final box
\NewDocumentCommand\final{ m } {\boxed{#1}}

% Custom math commands
\NewDocumentCommand\R{} {\mathbb{R}}
\ifxetex \RenewDocumentCommand\C{} {\mathbb{C}} \else \NewDocumentCommand\C{} {\mathbb{C}} \fi
\NewDocumentCommand\Z{} {\mathbb{Z}}
\NewDocumentCommand\Q{} {\mathbb{Q}}
\NewDocumentCommand\N{} {\mathbb{N}}
\NewDocumentCommand\F{} {\mathbb{F}}


% Custom units
\DeclareSIUnit{\mile}{mi}
\DeclareSIUnit{\gallon}{gallon}
\DeclareSIUnit{\pound}{lb}
\DeclareSIUnit{\foot}{ft}
\DeclareSIUnit{\atmosphere}{atm}
\DeclareSIUnit{\fahrenheit}{\ensuremath{{}^{\circ}
 \text{F}}}
\DeclareSIUnit{\atom}{at}
\DeclareSIUnit{\molecule}{molecule}
\DeclareSIUnit{\calorie}{cal}
\DeclareSIUnit{\Calorie}{Cal}
\DeclareSIUnit{\inch}{in}

% Unit vector formatting
\NewDocumentCommand\unit{ m } {\bm{\hat{\mathbf{#1}}}}

% Parts environment
\NewDocumentEnvironment{parts}{ }
		               {\begin{enumerate}[label=\textup{(\alph*)},
		                                  labelwidth=4.5em,
		                                  labelsep=0.5em,
		                                  itemsep=0em,
		                                  resume]}
		               {\end{enumerate}}

% \part[]
\RenewDocumentCommand\part{ o }{\IfNoValueTF{#1}
                                            {\item}
                                            {\item\label{#1}}}

%% Define Theorem-like environments
% QED Symbol
\RenewDocumentCommand\qedsymbol{}{$\blacksquare$}

% Make shaded theorems have uniform padding
\NewDocumentCommand\headpadding{}{\leftskip=0.4cm\rightskip=0.4cm\vspace{0.4cm}}
\NewDocumentCommand\footpadding{}{\vspace{0.4cm}}
% \NewDocumentCommand\theoremvert{}{0.6em}
\NewDocumentCommand\theoremvert{}{0em}

\makeatletter
\renewenvironment{proof}[1][\proofname]{\par
  \vspace{-\parskip}% remove the space after the theorem
  \pushQED{\qed}%
  \normalfont
  % \topsep0pt \partopsep0pt % no space before
  \topsep0.5\baselineskip \partopsep0pt
  \trivlist
  \item[\hskip\labelsep
        \itshape
    #1\@addpunct{.}]\ignorespaces
}{%
  \popQED\endtrivlist\@endpefalse
  % \addvspace{6pt plus 6pt} % some space after
  \vspace{-0.5\baselineskip}
}
\makeatother

% Solution (clone of proof)
\NewDocumentEnvironment{solution}{ o }
                       {\IfNoValueTF{#1}{\begin{proof}[Solution]}{\begin{proof}[#1]}}
                       {\end{proof}}

% Lemma style definition
\declaretheoremstyle[
	spaceabove=0.5\baselineskip, spacebelow=0em,
	headfont=\normalfont\bfseries,
	bodyfont=\normalfont\itshape
]{lemma}

\declaretheorem[style=lemma]{lemma}
\declaretheorem[style=lemma]{theorem}
\declaretheorem[style=lemma]{corollary}
\declaretheorem[style=lemma]{conjecture}

%% Definition Style Theorems
%%
%% Since Definitions are indented, we have 
%% to use a hack to get non-stupid spacing
%% and prevent footnotes from getting screwed up.
%%
%% This invloves creating separate document
%% environment wrappers for each theorem.

	% Create the definition style
	\declaretheoremstyle[
		headfont=\normalfont\bfseries,
		bodyfont=\normalfont,
		spaceabove=0em, spacebelow=0em,
	]{definition}

	% Create the theorem styles
	\declaretheorem[
		style=definition,
		title={Definition}
	]{cdefinition}
	\declaretheorem[
		style=definition,
		title={Notation},
		numbered=no
	]{cnotation}
	\declaretheorem[
		style=definition,
		title={Abuse~of~Notation},
		numbered=no
	]{cabuse}

	% Create environment wrappers for each theorem
	\NewDocumentEnvironment{definition}{o}{
		% Pre Content
		\begin{adjustwidth}{1cm}{1cm}
			\vspace*{-\baselineskip} % adjustwidth adds extra space
			\IfValueTF{#1}{
				\begin{cdefinition}[#1]
			}{
				\begin{cdefinition}
			}
			\IfValueT{#1}{\note{\textbf{#1}}}
	}{
		% Post Content
			\end{cdefinition}
		\end{adjustwidth}
	}

	\NewDocumentEnvironment{notation}{o}{
		% Pre Content
		\begin{adjustwidth}{1cm}{1cm}
			\vspace*{-\baselineskip} % adjustwidth adds extra space
			\IfValueTF{#1}{
				\begin{cnotation}[#1]
			}{
				\begin{cnotation}
			}
			
	}{
		% Post Content
			\end{cnotation}
		\end{adjustwidth}
	}

	\NewDocumentEnvironment{abuse}{o}{
		% Pre Content
		\begin{adjustwidth}{1cm}{1cm}
			\vspace*{-\baselineskip} % adjustwidth adds extra space
			\IfValueTF{#1}{
				\begin{cabuse}[#1]
			}{
				\begin{cabuse}
			}
			
	}{
		% Post Content
			\end{cabuse}
		\end{adjustwidth}
	}

%% End Definition Style Theorems

%% Block Style Theorems
%%
%% Like the definition style theorems,
%% we need to use a hack to get the
%% spacing and indentation right.

	% Declare block theorem style
	\declaretheoremstyle[
		headfont=\normalfont\bfseries,
		bodyfont=\normalfont,
	]{block}

	% Declare theorem environments
	\declaretheorem[
		style=block,
		postheadhook=\hspace{1em},
		title={Problem}
	]{cproblem}
	\declaretheorem[
		style=block,
		postheadhook=\hspace{1em},
		sibling=cproblem,
		title={Exercise}
	]{cexercise}

	% Create document environment wrappers for each theorem
	\NewDocumentEnvironment{problem}{o}{
		% Pre Content
		\begin{mdframed}[
			hidealllines=true,
			backgroundcolor={backgroundgrey},
			innertopmargin=0.6cm,
			innerleftmargin=0.4cm,
			innerrightmargin=0.4cm,
			innerbottommargin=0.4cm,
		]
			\IfValueTF{#1}{
				\begin{cproblem}[#1]
			}{
				\begin{cproblem}
			}
			
	}{
		% Post Content
			\end{cproblem}
		\end{mdframed}
		\vspace*{-0.5\baselineskip} % adjustwidth adds extra space
	}

	\NewDocumentEnvironment{exercise}{o}{
		% Pre Content
		\begin{mdframed}[
			hidealllines=true,
			backgroundcolor={backgroundgrey},
			innertopmargin=0.6cm,
			innerleftmargin=0.4cm,
			innerrightmargin=0.4cm,
			innerbottommargin=0.4cm,
		]
			\IfValueTF{#1}{
				\begin{cexercise}[#1]
			}{
				\begin{cexercise}
			}
			
	}{
		% Post Content
			\end{cexercise}
		\end{mdframed}
		\vspace*{-0.5\baselineskip} % adjustwidth adds extra space
	}

%% End block style theorems


%% Break Style Theorems
%%
%% Like the definition style theorems,
%% we need to use a hack to get the
%% spacing and indentation right.

	% Create break style
	\declaretheoremstyle[
		headfont=\normalfont\bfseries,
		bodyfont=\normalfont,
		postheadspace=\newline
	]{break}

	% Create break theorems
	\declaretheorem[
		style=break,
		title={Example}
	]{cexample}

	% Create document wrappers for each theorem
	\NewDocumentEnvironment{example}{o}{
		% Pre Content
		\begin{mdframed}[
			hidealllines=true,
			backgroundcolor={backgroundgold},
			innertopmargin=0.6cm,
			innerleftmargin=0.4cm,
			innerrightmargin=0.4cm,
			innerbottommargin=0.4cm,
		]
			\IfValueTF{#1}{
				\begin{cexample}[#1]
			}{
				\begin{cexample}
			}
			
	}{
		% Post Content
			\end{cexample}
		\end{mdframed}
		\vspace*{-0.5\baselineskip} % adjustwidth adds extra space
	}

%% End break style theorems